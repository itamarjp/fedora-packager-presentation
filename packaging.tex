\documentclass{beamer}

\usepackage[utf8]{inputenc}
\usepackage[T1]{fontenc}
\usepackage[portuges]{babel}

\usepackage{listings}
\usepackage{color}

\usetheme{Copenhagen}
\setbeamertemplate{headline}{}

\title{Tutorial de Empacotamento}
\author{Itamar Reis Peixoto - itamarjp@fedoraproject.org}
\institute{Fedora Project}
\date{2013}



\lstset{
  language=C,                % choose the language of the code
  numbers=left,                   % where to put the line-numbers
  stepnumber=1,                   % the step between two line-numbers.        
  numbersep=5pt,                  % how far the line-numbers are from the code
  backgroundcolor=\color{white},  % choose the background color. You must add \usepackage{color}
  showspaces=false,               % show spaces adding particular underscores
  showstringspaces=false,         % underline spaces within strings
  showtabs=false,                 % show tabs within strings adding particular underscores
  tabsize=2,                      % sets default tabsize to 2 spaces
  captionpos=b,                   % sets the caption-position to bottom
  breaklines=true,                % sets automatic line breaking
  breakatwhitespace=true,         % sets if automatic breaks should only happen at whitespace
  title=\lstname,                 % show the filename of files included with \lstinputlisting;
}








%% mostrar o plano no início de cada secção
\AtBeginSection[] 
{
\begin{frame}<beamer>
\frametitle{Temas Abordados}
\tableofcontents[currentsection]
\end{frame}
}


\begin{document}
\begin{frame}
% primeiro slide
\maketitle
\end{frame}

\begin{frame}
\frametitle{Temas Abordados}
\tableofcontents
\end{frame}



\section{Pre-Requisitos}
\begin{frame}
\frametitle{Pre-Requisitos}
\framesubtitle{ferramentas de desenvolvimento necessarias}
\begin{itemize}
\item yum -y install gcc
\item yum -y install automake autoconf
\item yum -y install rpmdevtools
\item yum -y install glibc-devel
\end{itemize}
\end{frame}






\section{Hello World}
\begin{frame}[fragile]
\frametitle{Hello World}
\framesubtitle{fad.c}
\begin{lstlisting}
#include <stdio.h>
int main(){
    printf("Fedora FAD SP 2013");
    return 0;
}
\end{lstlisting}
\end{frame}



\section{Manual do Aplicativo}
\begin{frame}
\frametitle{Manual do Aplicativo}
\framesubtitle{readme.txt}
escrever um manual descrevendo as funcionalidades do aplicativo e salvar como
readme.txt
\end{frame}


\section{gnu autoautotools}
\begin{frame}
\frametitle{gnu autoautotools}
\framesubtitle{Adicionando suporte para o GNU autotools em nosso aplicativo}

\begin{itemize}
\item rodar o autoscan
\item rodar o autoheader
\end{itemize}
\end{frame}


\section{gnu autoautotools}
\begin{frame}
\frametitle{gnu autoautotools}
\framesubtitle{Adicionando suporte para o GNU autotools em nosso aplicativo}


editar o arquivo configure.ac ajustando o nome do pacote e versao-> \\
e adicionar nele 
AM\_INIT\_AUTOMAKE \\


adicionar no Makefile.am ->  \\

bin\_PROGRAMS =  fad \\
fad\_SOURCES  = fad.c \\

apos isto rodar autoreconf -fiv 


\end{frame}







\section{criando o rpm}
\begin{frame}
\frametitle{criando o rpm}
\framesubtitle{criando o rpm}
rpmdev-newspec fad \\
adjustar o fad.spec 

rpmbuild -ba fad.spec (compila o fad.spec e gera o rpm + srpm) \\ 
rpmbuild -bs fad.spec (compila o fad.spec e gera apenas o srpm) \\ 


\end{frame}


\section{submetendo o rpm para repositorio oficial}
\begin{frame}
\frametitle{submetendo o rpm para repositorio oficial}
\framesubtitle{submetendo o rpm para repositorio oficial}
submetendo o rpm para repositorio oficial
\end{frame}


\section{Conclusão}
\begin{frame}
\frametitle{Conclusão}
\framesubtitle{Conclusão}
Conclusão
\end{frame}





\end{document}
