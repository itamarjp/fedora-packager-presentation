\documentclass[14pt]{beamer}
\usepackage[utf8]{inputenc}
\usepackage[T1]{fontenc}
\usepackage[portuges]{babel}
\usetheme{Copenhagen}

\title{Tutorial de Empacotamento}
\author{Itamar Reis Peixoto - itamarjp@fedoraproject.org}
\institute{Fedora Project}
\date{2013}


%% mostrar o plano no início de cada secção
\AtBeginSection[] 
{
\begin{frame}<beamer>
\frametitle{Plano}
\tableofcontents[currentsection]
\end{frame}
}


\begin{document}
\begin{frame}
% primeiro slide
\maketitle
\end{frame}

\begin{frame}
\frametitle{Plano}
\tableofcontents
\end{frame}



\section{Pre-Requisitos}

\begin{frame}
\frametitle{Indice}
% opcional

\framesubtitle{...}
% opcional
% conteúdo (LATEX normal)
\begin{itemize}
\item escrever slides sucintos;
\item usar um tamanho de letra grande;
\item respirar durante a apresentação;
\item repetir sempre as perguntas.
\end{itemize}
\end{frame}

\section{Hello World}

\section{gnu autoautotools}

\section{criando o rpm}

\section{submetendo o rpm para repositorio oficial}

\section{Conclusão}


\end{document}
